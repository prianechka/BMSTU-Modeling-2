\anonsection{Условие лабораторной работы}
Требуется реализовать ПО, позволяющее генерировать алгоритмическим методом последовательность случайных чисел и проверять итоговую последовательность на случайность по любому критерию.
Также нужно добавить пользователю возможность генерировать последовательность из однозначных чисел и реализовать генерацию табличным методом.

\anonsection{Теоретическая часть}
В этом разделе будет приведено описание методов генерации последовательности случайных чисел и описан критерий проверки последовательности на случайность.

\subsection*{Виды генераторов случайных чисел}
Всего можно выделить четыре типа генераторов случайных чисел:
\begin{enumerate}
	\item \textbf{Аппаратные генераторы} используют результаты определённых физических процессов для создания требуемой последовательности. Аппаратный генератор случайных чисел состоит из источника энтропии и устройства, преобразующего значения, полученные с источника энтропии, в нужный формат.
	
	К такому типу относятся генераторы, основанные на фотоэффекте или тепловом шуме при работе полупроводникового диода. 
	На выходе получается последовательность, обладающая значительной степенью случайности, но у таких генераторов есть два недостатка: системы трудно реализовать в жизни, а процессов, позволяющие преобразовать энтропию в последовательность.
	
	\item \textbf{Алгоритмические генераторы} основаны на фиксированных алгоритмах, которые, в зависимости от некоторых физических параметров (например, содержимого ввода/вывода), выдают нужный результат. Подобные алгоритмы имеют программную реализацию и используются в коммерческом ПО.
	
	\item \textbf{Табличные генераторы} принимают на вход уже готовую последовательность, обладающую свойством случайности, после чего проводит различные манипуляции с ней (комбинирование, перемешивание), и выдают результат.
	К недостаткам этого подхода можно отнести лишнее использование памяти, предопределённость значений и ограниченность последовательности.
\end{enumerate}

\subsection*{Выбранный алгоритмический метод}
В качестве алгоритмического метода генерации случайной последовательности выбран метод \textbf{линейной конгруэнтной последовательности}.

Для осуществления генерации чисел данным методом, необходимо задать 4 числа: модуль $m > 0$, множитель $a \in [0, m]$, приращение $c \in [0, m]$ и начальное число $x_0 \in [0, m]$ 

Последовательность случайных чисел генерируется рекуррентно при помощи формулы \ref{c}:
\begin{equation}
	\label{c}
	X_{n + 1} = (a * X_n + c) mod m
\end{equation}

Одним из требований к линейным конгруэнтным последовательностям является как можно большая длина периода. Длина периода зависит от значений $m, a и c$. 
Линейная конгруэнтная последовательность, определенная числами $m, a, c и x_0$, имеет период длиной M тогда и только тогда, когда:
\begin{itemize}
	\item числа $m$ и $c$ взаимно простые;
	\item $a-1$ кратно $p$ для каждого простого $p$, являющегося делителем $m$;
	\item $a-1$ кратно 4, если $m$ кратно 4.
\end{itemize}




\subsection*{Выбранный критерий определения случайности}
В качестве критерия был выбран критерий Бартерса.

Пусть $H_0$ -- гипотеза о том, что последовательность обладает свойством случайности. На этом этапе требуется выбрать также уровень значимости $\alpha$.
На основе выборки $X$ создаётся вариационный ряд $X_1$, а затем рассчитывается ряд $R$, где $R_i$ -- количество вхождений элемента в последовательность.

После предварительных расчётов вычисляется эмпирический коэффициент $B$ по формуле \ref{b}:
\begin{equation}
	\label{b}
	B = \frac{\sum_{i}^{n - 1} (R_i - R_{i + 1})^2}{\sum_{i}^{n} (R_i - R_{mean})^2}
\end{equation}

Затем, исходя из $\alpha$ и таблицы 1, требуется выбрать коэффициенты $a$, $b$, $c$, $d$. 

\FloatBarrier
\begin{table}[h]
	\label{t}
	\caption{Таблица выбора коэффициентов}
	\centering
	\begin{tabular}{ | l | l | l | l | l |}
		\hline
		$\alpha$ & 0.01 & 0.025 & 0.05 & 0.1 \\ 
		\hline
		a & -0.023 & -0.004 & 0.119 & -0.465 \\
		b & 0.261 & 0.381 & 0.440 & 1.184 \\
		c & -0.345 & -0.266 & -0.230 & -0.088 \\
		d & 2.212 & 1.748 & 1.520 & 0.674 \\
		\hline
	\end{tabular}
\end{table}
\FloatBarrier

Для того, чтобы определить, случайна ли последовательность, требуется рассчитать ещё один коэффициент $B_{\alpha}$ по формуле \ref{b1}
\begin{equation}
	\label{b1}
	B_{\alpha} = a+bn^c(ln n)^d
\end{equation}

Если коэффициент $B$ вошёл в интервал $[(2 - B_{\alpha}), 2 + B_{\alpha}]$, то принимается решение об истинности гипотезы $H_0$.

В качестве величины случайности было выбрано значение $ (B * (2 - B_\alpha)) * ((2 + B_\alpha) * B)$. Чем больше это число, тем выше величина случайности.

\newpage

\anonsection{Реализация}
В этом разделе будет приведены листинги кода реализации алгоритмов, продемонстрирована работа программы и построены таблицы с результатами.

\subsection*{Листинги кода}
Для реализации ПО был использован язык Python, так как был использована библиотека numpy, обладающая значительными возможностями в генерации случайных чисел и манипулирования большим объемом данных.

На листинге 1 представлена реализация алгоритма генерации табличной последовательности.

На листинге 2 представлена реализация алгоритма генерации алгоритмической последовательности. 

На листинге 3 представлена реализация критерия проверки последовательности на случайности.

\begin{lstinputlisting}[language=Python, caption=Реализация алгоритма генерации табличной последовательности, linerange={4-15}, 
	basicstyle=\footnotesize\ttfamily, frame=single,breaklines=true]{../src/main.py}
\end{lstinputlisting}
\FloatBarrier

\FloatBarrier
\begin{lstinputlisting}[language=Python, caption=Реализация алгоритма генерации алгоритмической последовательности, linerange={34-38}, 
	basicstyle=\footnotesize\ttfamily, frame=single, breaklines=true]{../src/main.py}
\end{lstinputlisting}
\FloatBarrier

\FloatBarrier
\begin{lstinputlisting}[language=Python, caption=Реализация критерия проверки последовательности на случайности, linerange={18-30}, 
	basicstyle=\footnotesize\ttfamily, frame=single, breaklines=true]{../src/main.py}
\end{lstinputlisting}
\FloatBarrier

\subsection*{Демонстрация работы программы}
В ходе выполнения программы пользователю доступно меню, в котором он может выбрать размер последовательности, который нужно сгенерировать, уровень значимости, а также файл, если он хочет также проверить собственную последовательность.

На рисунке представлена демонстрация главного меню программы:
\FloatBarrier
\begin{figure}[h]
	\begin{center}
		\includegraphics[width=\linewidth]{inc/run.png}
	\end{center}
	\caption{Демонстрация работы программы}
\end{figure}
\FloatBarrier

\subsection*{Полученные результаты}
Тестирование проводилось на последовательностях разной длины для каждого из трёх способов с уровнем значимости $\alpha = 0.1$. 
Полученные результаты представлены в таблицах 2-4: 

\FloatBarrier
\begin{table}[h]
	\caption{Таблица полученных значений для табличного способа}
	\centering
	\begin{tabular}{ | l | p{5cm} | l | l |}
		\hline
		Размер & Последовательность & Величина случайности & Случайна? \\ 
		\hline
		5 & 103 436 861 271 107 & -0.006 & Нет  \\ 
		\hline
		25 & 872 344 492 309 467  21 107 331 615 271 701 131 861 664 459  88 373 122
		100 770 215 103 662 436  72 & 1.207 & Да \\
		\hline
		50 & - & 1.831 & Да \\
		\hline
		100 & - & 2.354 & Да  \\
		\hline
		1000 & - & 1.748 & Да \\
		\hline
	\end{tabular}
\end{table}
\FloatBarrier

\FloatBarrier
\begin{table}[h]
	\caption{Таблица полученных значений для алгоритмического способа}
	\centering
	\begin{tabular}{ | l | p{5cm} | l | l |}
		\hline
		Размер & Последовательность & Величина случайности & Случайна? \\ 
		\hline
		5 & 7, 80, 883, 707, 773 & 0 & Да  \\
		\hline
		25 & 7, 80, 883, 707, 773, 498, 476, 234, 575, 322, 542, 960, 553, 80, 883, 707, 773, 498, 476, 234, 575, 322, 542, 960, 553 & -1.596 & Нет \\
		\hline
		50 & - & 0.503 & Да \\
		\hline
		100 & - & 0.241 & Да  \\
		\hline
		1000 & - & 0.09 & Да \\
		\hline
	\end{tabular}
\end{table}
\FloatBarrier

\FloatBarrier
\begin{table}[h]
	\caption{Таблица полученных значений для ручного способа}
	\centering
	\begin{tabular}{ | l | p{5cm} | l | l |}
		\hline
		Размер & Последовательность & Величина случайности & Случайна? \\ 
		\hline
		5 & 1, 7, 9, 5, 7 & -0.006 & Нет  \\
		\hline
		25 & 1, 2, 9, 3, 3, 5, 4, 8, 9, 1, 3, 8, 1, 5, 4, 9, 5, 1, 2, 3, 6, 9, 2, 3, 5  & 0.382 & Да \\
		\hline
		50 & - & -0.212 & Нет \\
		\hline
	\end{tabular}
\end{table}
\FloatBarrier

\subsection*{Выводы}
В случае малого размера последовательности величина случайности последовательности близка к нулю, поэтому выводы можно сделать только на большом объёме последовательности.
Табличный метод показал самые высокие показатели, все последовательности которой прошли порог, начиная с $N = 25$. 
В случае алгоритмического метода на $N = 25$ получилось не случайная последовательности, и на других $N$ величина случайности оказалась меньше, чем в табличном методе.
Проблема связана с тем, что по условию задачи нельзя генерировать числа, большие 1000, соответственно, под такой маленький диапазон нет подходящих параметров под метод линейной конгруэнтной последовательности.